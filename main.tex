\documentclass{article}
\usepackage[T1]{fontenc}

\usepackage{amsmath}

% Language setting
% Replace `english' with e.g. `spanish' to change the document language
\usepackage[polish,english]{babel}
\selectlanguage{polish}

% Set page size and margins
% Replace `letterpaper' with `a4paper' for UK/EU standard size
\usepackage[letterpaper,top=2cm,bottom=2cm,left=3cm,right=3cm,marginparwidth=1.75cm]{geometry}


% Useful packages
\usepackage{bm}
\usepackage{mathtools}
\usepackage{graphicx}
\usepackage{MnSymbol}
\usepackage[colorlinks=true, allcolors=blue]{hyperref}

\title{Projekt podstawy podejmowania decyzji - Hitori solver}
\author{Andrzej Dąbrowski}

\begin{document}
\maketitle

\section{Wprowadzenie}

Hitori to logiczna gra łamigłówkowa, która została stworzona w Japonii prze firmę Nikoli. Celem gry jest wykreślanie liczb na planszy w taki sposób, aby żadna liczba nie powtarzała się w wierszu i kolumnie, a także aby żadne dwie wykreślone liczby nie były sąsiadującymi polami - zarówno w poziomie, jak i w pionie.

Pełna lista zasad Hitori:
\begin{enumerate}
    \item Żadna liczba nie może się powtarzać w wierszu lub kolumnie.
    \item Żadne dwie takie same liczby nie mogą być sąsiadującymi polami (dotykać się bokami).
    \item Wszystkie puste pola muszą być dostępne z pozostałymi polami planszy poprzez ruchy w pionie lub poziomie. Nie mogą istnieć odizolowane grupy pustych pól.
\end{enumerate}

Podczas rozwiązywania Hitori trzeba analizować układ liczb na planszy i strategicznie wykreślać odpowiednie liczby, aby spełnić powyższe zasady. Rozwiązanie jest zawsze unikalne dla każdej planszy.

\section{Rozwiązanie}
\subsection{Dane:}

% unique numbers
$ u $ - liczba unikalnych cyfr zapisabnych w tablicy gry \\
$ j $ - wymiary tablicy gry \\

% board 
$ D_{j\times j} $ - macierz reprezentująca plansze gry

% board state
$ B_{j\times j \times u} $ 
% - macierz 3d reprezentująca plansze gry w postaci binarnej
- lista macierzy binarych; w jedenej macierzy zakodowano miejsce występowania danej liczby w tablicy gry.

% decision
$ X_{3j\times 3j} = 
\begin{bmatrix}
x_{1 1} & \cdots & x_{1 3j}\\
\vdots & \ddots & \vdots\\
x_{3j 1} & \cdots & x_{3j 3j}
\end{bmatrix}$
- binarna macierz decyzji, gdzie 1 - pole zacieniowane, 0 - pole nie zacieniowane. 
\\
\\
Macierz $X$ ma wymiary 3 razy większe od wymiarów macierzy $B$ z powodu implementacji ograniczenia związanego z zasadą nr.3 w CPLEX. $j$ pierwszych i $j$ ostatnich kolumn oraz wierszy stanową kolumny pomocnicze, które są zapełnione wartościami 1. Kolumny i wiersze od $j$ do $2j$ stanowią aktualną macierz decyzji o zacieniowaniu komórki.
\\
% cost function
\subsection{Funkcja celu:}
\begin{equation}
F(X) = \sum_{n=1}^{3j}\sum_{i=1}^{3j} x_{in}
\end{equation} - całkowite pokrycie zacieniowanymi polami

% target function
\begin{equation}
X^* = \arg \min_{X} \sum_{n=1}^{3j}\sum_{i=1}^{3j} x_{in}
\end{equation} - minimalne całkowite pokrycie zacieniowanymi polami \\
\subsection{Ogranicznia:}
% Ones_In_Left_Support_Columns
Dla każdego y $\in\{1, 2, ..., j\}$:
\begin{equation}
\sum_{i=1}^{3j} x_{iy} = 3j
\end{equation} - ograniczenie związane z zapełnieniem lewych kolumn pomocniczych wartością 1
\\
% Ones_In_Right_Support_Columns
Dla każdego y $\in\{1, 2, ..., j\}$:
\begin{equation}
\sum_{i=1}^{3j} x_{iy} = 3j
\end{equation} - ograniczenie związane z zapełnieniem prawych kolumn pomocniczych wartością 1
\\
% Ones_In_Upper_Support_Rows
Dla każdego w $\in\{1, 2, ..., j\}$:
\begin{equation}
\sum_{i=1}^{3j} x_{wi} = 3j
\end{equation} - ograniczenie związane z zapełnieniem górnych rzędów pomocniczych wartością 1
\\
% Ones_In_Bottom_Support_Rows
Dla każdego w $\in\{1, 2, ..., j\}$:
\begin{equation}
\sum_{i=1}^{3j} x_{wi} = 3j
\end{equation} - ograniczenie związane z zapełnieniem dolnych rzędów pomocniczych wartością 1
% Repeating_Numbers_in_Rows
\\
Dla każdego f $\in\{1, 2, ..., u\}$: 
    oraz dla każdego w $\in\{1, 2, ..., j\}$:
        \begin{equation}
        \sum_{i=2j}^{3j} (1-x_{wi})*b_{fwi} \leq 1
        \end{equation} - ograniczenie związane z powtarzającymi się liczbami w tablicy gry w rzędach
\\
% Repeating_Numbers_in_Columns
Dla każdego f $\in\{1, 2, ..., u\}$:
    oraz dla każdego y $\in\{1, 2, ..., j\}$:
        \begin{equation}
        \sum_{i=2j}^{3j} (1-x_{iy})b_{fiy} \leq 1
        \end{equation} - ograniczenie związane z powtarzającymi się liczbami w tablicy gry w kolumnach
\\
% Adjacent_Black_by_Row
Dla każdego r $\in\{1, 2, ..., j\}$:
    oraz dla każdego y $\in\{1, 2, ..., j - 1\}$:
        \begin{equation}
        x_{ry} + x_{r(y+1)} \leq 1
        \end{equation} - ograniczenie związane z eliminacją sąsiadujących ze sobą zacieniowanych pól w wierszach
\\
% Adjacent_Black_by_Column
Dla każdego c $\in\{1, 2, ..., j\}$:
    oraz dla każdego w $\in\{1, 2, ..., j-1\}$:
        \begin{equation}
        x_{wc} + x_{(w+1)c} \leq 1
        \end{equation} - ograniczenie związane z eliminacją sąsiadujących ze sobą zacieniowanych pól w kolumnach
\\
% Closed_Shapes
Dla każdego y $\in\{1, 2, ..., j - 1\}$
    , dla każdego w $\in\{j+1, j+2, ..., 2j\}$
        oraz dla każdego n $\in\{j+1, j+2, ..., 2j\}$
        \begin{equation}
            x_{(y-1+p)(w+1+n-p)} \\
            + x_{(y-1+p)(w-1-n+p)} \\
            + x_{(y+1-p)(w+1+n-p)} \\
            + x_{(y+1-p)(w-1-n+p)} \leq 4n+ 4 - 1
            \end{equation} - ograniczenie związane z zasadą numer 3.

Powyższe ograniczenie ma pewne restrykcje i nie jest w stanie wykrywać pewnych zakreśleń łamiących zasadę nr.3. Jest to związane z ograniczeniami solvera CPLEX. Przykładowe zacieniowanie nie wykrywane przez wspomniane ograniczenie:

$$ \begin{bmatrix}
3 & 3 & \blacksquare & 3 & 3 \\
3 & 3 & 3 & \blacksquare & 3 \\
3 & 3 & \blacksquare & 3 & 3 \\
3 & 3 & 3 & \blacksquare & 3 \\
3 & 3 & \blacksquare & 3 & 3 
\end{bmatrix}  $$

\section{Wyniki}

Dla u = 5 i j = 5 oraz planszy

$$ D_{5\times 5} = \begin{bmatrix}
2 & 2 & 1 & 5 & 3 \\
2 & 3 & 1 & 4 & 5 \\
1 & 1 & 1 & 3 & 5 \\
1 & 3 & 5 & 4 & 2 \\
5 & 4 & 3 & 2 & 1 
\end{bmatrix}  $$

Poprawne rozwiązanie:
$$ \begin{bmatrix}
\blacksquare & 2 & \blacksquare & 5 & 3 \\
2 & 3 & 1 & 4 & \blacksquare \\
\blacksquare & 1 & \blacksquare & 3 & 5 \\
1 & \blacksquare & 5 & \blacksquare & 2 \\
5 & 4 & 3 & 2 & 1 
\end{bmatrix}  $$

Solver zwrócił wynik (zmienne pomocnicze pominięto):

$$ X^{*} = \begin{bmatrix}
1 & 0 & 1 & 0 & 0 \\
0 & 0 & 0 & 0 & 1 \\
1 & 0 & 1 & 0 & 0 \\
0 & 1 & 0 & 1 & 0 \\
0 & 0 & 0 & 0 & 0 
\end{bmatrix}  $$
Wizualizacja:
$$ \begin{bmatrix}
\blacksquare & 2 & \blacksquare & 5 & 3 \\
2 & 3 & 1 & 4 & \blacksquare \\
\blacksquare & 1 & \blacksquare & 3 & 5 \\
1 & \blacksquare & 5 & \blacksquare & 2 \\
5 & 4 & 3 & 2 & 1 
\end{bmatrix}  $$

Solver poprawnie zacieniował 7 komórek, przez co rozwiazał testową instancję puzzli Hitori poprawnie.

Do drugiego rozwiązania użyto metaheurytsyki Tabu Search o długości tablicy Tabu = 10 oraz maksymalnej ilości iteracji 1000.

Algorytm uruchomiono 20 razy. Dla żadnego z uruchomień algorytm nie wyznaczył wszystkich 7 komórek do zacieniowania.

Statystyki zacieniowanych komórek: \\
Średnia: 1.2 \\
Mediana: 1.0 \\
Max: 3.0 \\
Min: 0.0 \\
Odchylenie standardowe: 0.89 \\

Przykładowe rozwiązanie wyznaczone przez Tabu Search  (zmienne pomocnicze pominięto):

$$ X^{*} = \begin{bmatrix}
1 & 0 & 0 & 0 & 0 \\
0 & 0 & 0 & 0 & 0 \\
1 & 0 & 1 & 0 & 0 \\
0 & 0 & 0 & 0 & 0 \\
0 & 0 & 0 & 0 & 0 
\end{bmatrix}  $$
Wizualizacja:
$$ \begin{bmatrix}
\blacksquare & 2 & 1 & 5 & 3 \\
2 & 3 & 1 & 4 & 5 \\
\blacksquare & 1 & \blacksquare & 3 & 5 \\
1 & 3 & 5 & 4 & 2 \\
5 & 4 & 3 & 2 & 1 
\end{bmatrix}  $$

% \section{Wnioski}
% Rozwiązanie wykorzystujące metaheurytykę, nie daje dobrych wyników w porównaniu do wyników uzyskiwanych przez solver.

\section{Wnioski}
Próbę rozwiązania puzzli Hitori przeporowadzono na dwa sposoby, poprzez solver oraz metaheurystykę. 
Przy użyciu solvera udało się osiągnąć poprawne rozwiązanie (rozwiązanie działa poprawnie z wyłączeniem pewnych szczególnych przypadków), rozwiązanie przy użyciu metaheurystyki nie osiągneło zadawalających wyników.
\\
\\
Link do kodów źródłowych rozwiązania:
\url{https://github.com/11jolek11/Hitori}

\end{document}
